\section{Partie Théorique}
\subsection{Borne inférieure théorique}
\noindent
Les performances du parcours en largeur pour appareiller les tétraèdres en diamants nous semblent raisonnables. Cependant, il serait bien de pouvoir assurer une borne minimum théorique pour l'appareillage. A priori, aucune borne minimale intéressante ne peut être apportée. En effet, certains maillages comme des chaines de tétraèdres où les tétraèdres sont juste reliés deux à deux n'autorisent pas le regroupement en diamant.\\
Néanmoins, nous allons dans cette partie essayer de dégager une borne inférieure sous certaines assomptions. Nous allons d'abord transformer notre maillage M en graphe annexe $G'(V',E')$. Dans ce graphe annexe, chaque point est une arête centrale potentielle du maillage M et deux points sont connectés si leurs arêtes correspondantes dans M appartiennent au même tétraèdre.\\
$V' = \{e\in E : e\notin B\}$ et $E'=\{(e,e'), e,e'\in E : e,e' \in T_i, \,  i\in \{0,|T|-1\}\}$.
\begin{itemize}
\item $V$ est l'ensemble des sommets de notre maillage M
\item $E$ est l'ensemble des arêtes de notre maillage M
\item $B$ est le bord du maillage M
\item $b$ est la part d'arêtes situées sur les bords dans le maillage M
\item $V'$ est l'ensemble des sommets de notre graphe $G'$
\item $E'$ est l'ensemble des arêtes de notre graphe $G'$
\item $d'$ est le degré maximum d'un sommet dans notre graphe $G'$
\end{itemize}

\begin{equation}
\label{eq:V}
|V'|=(1-b)\cdot |E|
\end{equation}
\begin{equation}
\label{eq:d}
|d'|= 27
\end{equation}
L'équation Eq. \ref{eq:V} vient du fait que seules les arêtes n'étant pas sur les bords du maillages peuvent être utlisées comme arête centrale et donc devenir des sommets dans notre graphe annexe $G'$.\\
L'équation Eq. \ref{eq:d} est la conséquence de notre choix d'appariement des tétraèdres en diamants contenant au maximum 9 tétraèdres. Un diamant contenant 9 tétraèdres contient 27 arêtes. Par conséquent (dans le pire des cas) chaque arête n'étant pas sur les bords du maillage M peut être adjacente à 9 tétraèdres. Le sommet correspondant dans notre graphe annexe sera alors adjacent à 27 autres sommets.\\
Calculer un Stable Maximum $\alpha(G')$ (Maximum Independant Set en anglais) dans ce graphe annexe $G'$ revient ainsi à choisir un ensemble d'arêtes maximal tel-que deux arêtes n'appartiennent pas au même tétraèdre. C'est donc une autre formulation de notre problème d'appariement des tétraèdres en diamants. Si on applique un algorithme glouton choisissant à chaque itération le sommet de maximum degré alors :
\begin{equation}
\alpha(G')\geqslant \frac{|V'|}{d+1} = \frac{(1-b)\cdot |E|}{28}
\end{equation}
On peut comparer cette borne inférieure aux résultats trouvés par notre parcours en largeur (Tab. \ref{tab:borne_inf_theorique}). On note que la borne inférieure est en moyenne 4 fois plus faible que le résultat trouvé par le parcours en largeur.
\begin{table}[h]
\footnotesize
\centering
\begin{tabular}{| c | c | c | c | c|}
\hline
Nom de la & Part d'arêtes & Nombre d'arêtes& Nombre d'arêtes centrales & Nombre d'arêtes centrales\\
structure & sur les bords du maillage & &choisies par le & minimum théorique\\
& & & parcours en largeur & \\
\hline
B1 & 0.44 & 5452 & 504 & 109 \\
B2 & 0.08 & 101343 & 12866 & 3329\\
B3 & 0.02 & 2647808 & 351997 & 92673 \\
B4 & 0.01 & 9575510 & 1281894 & 338563  \\
V1 & 0.27 & 182170 & 19942 & 4749 \\
M1 & 0.26 & 169115 & 18215 & 4469 \\
C1 & 0.33 & 217974 & 22919 & 5215\\
\hline  
\end{tabular}
\caption{Comparaison du nombre d'arêtes centrales choisies par le parcours en largeur et de la borne inférieure théorique}
\label{tab:borne_inf_theorique}
\end{table}


%\subsection{NP-Complétude}
%\noindent
%La pierre angulaire de notre structure est la constuction des diamants. Si tous les tétraèdres étaient regroupés en diamants, alors nous achèverions un encodage de la connectivité avec 2 rpt. Le parcours en largeur est satisfaisant mais existe-il un algorithme optimal en temps polynomial ? Nous allons montrer dans cette section que le problème est en réalité NP-dur.\\
%Notre problème est le suivant : \textbf{Existe-il une couverture des tétraèdres en diamants contenant plus de k diamants ?}\\
%Pour cela, nous allons procéder en 4 étapes :
%\begin{enumerate}
%\item Montrer que notre problème est NP
%\item Transformer notre maillage tétraèdrique en graphe en temps polynomial
%\item Résoudre MIS sur ce nouveau graphe
%\item Montrer que si l'on a une solution de MIS sur ce graphe alors on a une solution pour notre problème.
%\end{enumerate}
%Pour rappel, notre problème consiste à choisir un ensemble $E'$ d'arêtes tel-que deux arêtes de $E'$ n'appartiennent pas au même tétraèdre.\\
%1- Commencons par montrer que notre problème appartient à NP. Une solution est un ensemble d'arêtes (les arêtes centrales des diamants). Il suffit de vérifier pour chaque arête que A COMPLETER
%2- On peut créer un autre graphe où chaque arête du graphe initial est modélisée par un sommet. Deux sommets dans ce nouveau graphe sont connectées si et seulement si leurs arêtes dans le graphe original appartiennent au même tétraèdre. La création de ce nouveau graphe se fait en temps polynomial (en fonction des arêtes).\\
%3- Calculer un Stable Maximum (Maximum Indepent Set ou MIS en anglais) sur ce nouveau graphe signifie trouver un ensemble de sommets non connectés dans le graphe. C'est équivalent à trouver un ensemble d'arêtes n'appartenant pas aux mêmes tétraèdres dans notre maillage.\\
%4- Ainsi, calculer un MIS sur ce nouveau graphe est similaire à trouver un ensemble d'arêtes (et donc de diamants) indépendants.\\
%5- Par conséquent, trouver une solution à notre problème signifierait trouver une solution au MIS\\
%Nous avons donc réduit notre problème au MIS. Notre problème est donc NP-Dur.